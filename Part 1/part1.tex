\documentclass[../notesdecours.tex]{subfiles}

\begin{document}
    \part{Fondements de la Mécanique Quantique}

    \chapter{Postulats de la Mécanique Quantique}

    Dans ce chapitre, nous allons énoncer les postulats de la mécanique quantique. Ils permettront de répondre à des questions important dans l'élaboration de la théorique quantique; à savoir:
    \begin{enumerate}
    \item Comment décrire mathématiquement l'état d'un système quantique à un instant donné?
    \item Comment, cet état étant donné, prévoir les résultats de mesure des diverses grandeurs physiques?
    \item Comment trouver l'état du système à un instant t quelconque lorsqu'on connait  ce état à l'instant $t_0$?
    \end{enumerate}

    \section{Premier postulat : description des états quantiques $\ket{\psi}$}

    Dans la description classique de la Mécanique, on définit un espace de phase $M$, de dimension $2n$, $\forall n\in\mathbb{Z}$; permettant ainsi d'introduire les coordonnées $\left(q^i,p_j\right)$ liées par le crochet de poisson fondamental $\{q^i,p_j\}=\delta^i_j$. Un état est caractérisé par un point P dans l'espace des phases.\\

    En Mécanique Quantique, on définit un état comme une raie dans un espace de Hilbert $\mathcal{H}$. En introduisant la relation d'équivalence
    \begin{equation}
    \label{eq:equivalence}
        \ket{\psi} \sim \ket{\phi} \iff \exists \alpha\in\mathbb{C}\setminus\{0\} \text{ tel que } \ket{\psi}=\alpha\ket{\phi} ;
    \end{equation}
    nous pouvons définir l'espace des états comme étant $\mathcal{H}\setminus\ref{eq:equivalence}$. Les raies sont les classes d'équivalence, et $\ket{\psi}$ et $\ket{\phi}$ représentent le même état.\\

    \begin{property}
        Puisque $\mathcal{H}$ est un espace vectoriel, nous avons que pour tout $\ket{\psi},\ket{\phi}\in\mathcal{H}$, 
        \begin{equation}
            \alpha\ket{\psi}+\beta\ket{\psi}=\ket{\varphi}
        \end{equation}
        est également un état de $\mathcal{H}$. Il s'agit du principe de superposition : toute combinaison linéaire d'état donne un autre état dans la base de $\mathcal{H}$ilbert.
        \label{Superposition}
    \end{property}

    \section{Second postulat: observables $\ket{\hat{P}}$}

    \begin{definition}
        Un opérateur est hermitien lorsque $A=A^\dagger$.
    \end{definition}

    \begin{definition}
        Un opérateur hermitien est appelée observable lorsqu'il possède une base de vecteurs propres.
    \end{definition}

    \begin{theorem}
        Soit A une matrice hermitienne. Alors, il existe une base orthonormale $\{\ket{\lambda_i}\}$ telle que $A\ket{\lambda_i}=\lambda_i\ket{\lambda_i}$ pour toute valeure propre $\lambda_i\in\mathbb{R}$.
    \end{theorem}

    \begin{remark}
        Cet énoncé classique d'algèbre linéaire porte le nom de théorème spectral.
    \end{remark}

    \begin{remark}
        La mesure de l'observable A donne une valeur propre $\lambda_i$.
    \end{remark}

    Nous avons maintenant appris comment décrire un état quantique \textit{à un instant donné} et la version quantique d'une quantité observable. Il nous reste à développer la notion de mesure et d'évolution des états. C'est par ce dernier point que nous allons commencer, en énoncant le postulat d'évolution.

    \section{Troisième postulat: Evolution dans le temps}

    \begin{definition}
        L'évolution dans le temps est donné par un groupe unitaire à un paramètre $U(t)$, respectant les deux propriétés suivantes:
        \begin{itemize}
            \item $U(t)$ est un opérateur unitaire pour tout t ;
            \item $U(t_1)U(t_2)=U(t_1+t_2)$.
        \end{itemize}
        Un tel opérateur $U(t)$ est appelé \emph{opérateur d'évolution}.
    \end{definition}

    \begin{property}
        Si $\ket{\psi}$ est l'état du système à un instant $t_0$ donné, alors l'état à un temps $t>t_0$ sera donné par 
        \begin{math}
            \ket{\psi}(t) = \hat{U}(t-t_0)\ket{\psi}
        \end{math}
    \end{property}

    \begin{property}
        \label{Operateur d'evolution}
        Nous pouvons toujours écrire l'opérateur d'évolution sous la forme $U(t) = e^{-\frac{iH(t-t_0)}{\hbar}}\ket{\psi}$. 
    \end{property}

    \begin{definition}
        L'opérateur H que nous avons fais apparaître en \ref{Operateur d'evolution} est appelé Hamiltonien. 
    \end{definition}

    En particulier, nous retrouvons de cette égalité l'équation de Schrödinger. En effet,
    \begin{align}
        \frac{d\ket{\psi}}{dt} (t)= -\frac{iH}{\hbar}e^{-\frac{i(t-t_0)H}{\hbar}}\ket{\psi}(t=0) &= -\frac{iH}{\hbar}\ket{\psi}(t)\\
        i\hbar\frac{d\ket{\psi}}{dt} (t) &= H\ket{\psi}(t) \label{Schrodinger}
    \end{align}

    \begin{remark}
        L'équation d'évolution est linéaire ; elle est donc en particulier compatible avec le principe de superposition du premier postulat.
    \end{remark}

    Arrêtons-nous l'espace d'un instant pour nous poser une question. Jusque là, nous avons vu des équations linéaires en le temps: mieux, nous faisons face à des équations simples à résoudre. En partant de ce constat, il serait légitime de se demander comment; à partir d'équations linéaires, émergent les propriétés hautement non-linéaires rencontrées parfois en mécanique classique. Un bon exemple est les systèmes chaotiques.\\

    Il se trouve que toute la complexité d'un système est encodée dans le spectre de l'Hamiltonien du système.

    \section{Quatrième postulat: Interprétation de Bohr}

\end{document}
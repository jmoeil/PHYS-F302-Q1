\documentclass[11pt,twoside,a4paper]{report}

\usepackage[T1]{fontenc}
\usepackage[utf8]{inputenc}
\usepackage[french]{babel}
\usepackage{amsmath,amsfonts,amsthm,amssymb}
\usepackage{bm}
\usepackage{mathrsfs}
\usepackage{calrsfs}
\usepackage{wasysym}
\usepackage{dsfont}
\usepackage{caption}
\usepackage{multirow}
\usepackage{physics}
\usepackage{siunitx}
\usepackage{tikz}
\usepackage[allcolors=true]{hyperref}
\usepackage{cleveref}
\usepackage{pgfplotstable}
\usepackage{wrapfig}
\usepackage{graphicx}
\usepackage{subfiles}
\usepackage{systeme}
\usepackage{enumitem}
\usepackage{xcolor}
\usepackage{titlesec}
\usepackage{lmodern}
\usepackage{chngcntr}
\usepackage{textcomp}
\usepackage{geometry}
\geometry{top=3cm, bottom=3cm, left=2.5cm, right=2.5cm}
\pgfplotsset{width=10cm,compat=1.16}

% Counter stuff
\numberwithin{equation}{part}
\counterwithin*{section}{part}
\setcounter{tocdepth}{2}

% Random utilities...
\newcommand{\ti}{\ensuremath{\times}}
\newcommand{\h}{\ensuremath{\hbar}}
\newcommand{\med}{\ensuremath\cdot}

\renewcommand{\d}{\mathrm{d}}
\renewcommand{\thepart}{\arabic{part}}

% Special formatting for paragraphs
\titleformat{\paragraph}
  {\normalfont\normalsize\bfseries}
  {\theparagraph}
  {1em}
  {}
\titlespacing*{\paragraph}
  {0pt}
  {3.25ex plus 1ex minus .2ex}
  {1.5ex plus .2ex}

% Defining new theorem environments
\newtheorem{theorem}{Théorème}[section]
\newtheorem{definition}[theorem]{Définition}
\newtheorem{lemma}[theorem]{Lemme}
\newtheorem{property}[theorem]{Proposition}
\newtheorem{corollary}[theorem]{Corollaire}
\newtheorem{remark}[theorem]{Remarque}
\newtheorem{reminder}[theorem]{Rappel théorique}
\newtheorem{example}[theorem]{Exemple}

\renewcommand{\proofname}{Preuve}
\renewcommand{\qedsymbol}{\ensuremath{\blacksquare}}


%%% Début ajouts Sami
% Definition boite grise avec couleur grise définie
\definecolor{BGgris}{RGB}{222,230,230}
\newcommand\bg[2]{%
  \begin{center}%
    \fcolorbox{white}{BGgris}{%
      \parbox{.9\linewidth}{%
        \begin{large} \textit{#1} \end{large} \\%
        #2%
      }%
    }%
  \end{center}%
}

% Boite d'alerte
\definecolor{BGorange}{RGB}{255, 216, 154}
\newcommand\probleme[1]{%
  \begin{center}%
    \fcolorbox{white}{BGorange}{%
      \parbox{.9\linewidth}{%
        \begin{large} \textbf{Problème} \end{large} \\%
        #1%
      }%
    }%
  \end{center}%
}

\setlength{\fboxsep}{2em} % espace entre le bord d'une boite et le texte dedans

%%% Fin ajouts Sami

\title{%
  \textbf{PHYS-F302 - Mécanique Quantique} \\
  \textit{Basé sur le cours théorique du Prof. \href{mailto:frank.ferrari@ulb.be}{Frank Ferrari}}
}
\author{%
  \href{mailto:juian.moeil@ulb.be}{Moeil Juian} \\
  \href{mailto:hallemans.alexandre@ulb.be}{Hallemans Alexandre}
}
\date{\textbf{Année académique 2021-2022}}


\begin{document}
\maketitle

\begin{abstract}
    Ces notes sont basées sur les cours et les notes de Frank Ferrari. Dans la mesure où elles sont rédigées par des étudiant\med{}e\med{}s, il est probable qu'elles contiennent des fautes. N'hésitez pas à les signaler à \href{mailto:juian.moeil@ulb.be}{Juian Moeil} ou à \href{mailto:hallemans.alexandre@ulb.be}{Hallemans Alexandre} lorsque vous en trouvez.
\end{abstract}
\newpage

\part{Fondements}

\chapter{Postulats de la Mécanique Quantique}

Dans ce chapitre, nous allons énoncer les postulats de la mécanique quantique. Ils permettront de répondre à des questions important dans l'élaboration de la théorique quantique; à savoir:
\begin{enumerate}
\item Comment décrire mathématiquement l'état d'un système quantique à un instant donné?
\item Comment, cet état étant donné, prévoir les résultats de mesure des diverses grandeurs physiques?
\item Comment trouver l'état du système à un instant t quelconque lorsqu'on connait  ce état à l'instant $t_0$?
\end{enumerate}

\section{Premier postulat : description des états quantiques}

Dans la description classique de la Mécanique, on définit un espace de phase $M$, de dimension $2n$, $\forall n\in\mathbb{Z}$; permettant ainsi d'introduire les coordonnées $\left(q^i,p_j\right)$ liées par le crochet de poisson fondamental $\{q^i,p_j\}=\delta^i_j$. Un état est caractérisé par un point P dans l'espace des phases.\\

En Mécanique Quantique, on définit un état comme une raie dans un espace de Hilbert $\mathcal{H}$. En introduisant la relation d'équivalence
\begin{equation}
\label{eq:equivalence}
    \ket{\psi} \sim \ket{\phi} \iff \exists \alpha\in\mathbb{C}\setminus\{0\} \text{ tel que } \ket{\psi}=\alpha\ket{\phi} ;
\end{equation}
nous pouvons définir l'espace des états comme étant $\mathcal{H}\setminus\ref{equivalence}$. Les raies sont les classes d'équivalence, et $\ket{\psi}$ et $\ket{\phi}$ représentent le même état.\\

\begin{property}
    Puisque $\mathcal{H}$ est un espace vectoriel, nous avons que pour deux états $\ket{\psi},\ket{\phi} \in\mathcal{H}$, 
    \begin{equation}
        \alpha\ket{\psi}+\beta\ket{\psi}=\ket{\varphi}
    \end{equation}
    est également un état de $\mathcal{H}$. Il s'agit du principe de superposition : toute combinaison linéaire d'état donne un autre état dans la base de $\mathcal{H}$ilbert.
    \label{Superposition}
\end{property}

\section{Second postulat: observables $\ket{\hat{P}}$}

\begin{definition}
    Un opérateur est hermitien lorsque $A=A^\dagger$.
\end{definition}

\begin{definition}
    Un opérateur hermitien est appelée observable lorsqu'il possède une base de vecteurs propres.
\end{definition}

\begin{theorem}
    Soit A une matrice hermitienne. Alors, il existe une base orthonormale $\{\ket{\lambda_i}\}$ telle que $A\ket{\lambda_i}=\lambda_i\ket{\lambda_i}$ pour toute valeure propre $\lambda_i\in\mathbb{R}$.
\end{theorem}

\begin{remark}
    Cet énoncé classique d'algèbre linéaire porte le nom de théorème spectral.
\end{remark}

\begin{remark}
    La mesure de l'observable A donne une valeur propre $\lambda_i$.
\end{remark}

Nous avons maintenant appris comment décrire un état quantique \textit{à un instant donné} et la version quantique d'une quantité observable. Il nous reste à développer la notion de mesure et d'évolution des états. C'est par ce dernier point que nous allons commencer, en énoncant le postulat d'évolution.

\section{Troisième postulat: Evolution dans le temps}

\begin{definition}
    L'évolution dans le temps est donné par un groupe unitaire à un paramètre $U(t)$, respectant les deux propriétés suivantes:
    \begin{itemize}
        \item $U(t)$ est un opérateur unitaire pour tout t ;
        \item $U(t_1)U(t_2)=U(t_1+t_2)$.
    \end{itemize}
    Un tel opérateur $U(t)$ est appelé \emph{opérateur d'évolution}.
\end{definition}

\begin{property}
    Si $\ket{\psi}$ est l'état du système à un instant $t_0$ donné, alors l'état à un temps $t>t_0$ sera donné par 
    \begin{math}
        \ket{\psi}(t) = \hat{U}(t-t_0)\ket{\psi}
    \end{math}
\end{Property}

\begin{property}
    \label{Operateur d'evolution}
    Nous pouvons toujours écrire l'opérateur d'évolution sous la forme $U(t) = e^{-\frac{iH(t-t_0)}{\hbar}}\ket{\psi}$. 
\end{property}

\begin{definition}
    L'opérateur H que nous avons fais apparaître en \ref{Operateur d'evolution} est appelé Hamiltonien. 
\end{definition}

En particulier, nous retrouvons de cette égalité l'équation de Schrödinger. En effet,
\begin{align}
    \frac{d\ket{\psi}}{dt} (t)= -\frac{iH}{\hbar}e^{-\frac{i(t-t_0)H}{\hbar}}\ket{\psi}(t=0) &= -\frac{iH}{\hbar}\ket{\psi}(t)\\
    i\hbar\frac{d\ket{\psi}}{dt} (t) &= H\ket{\psi}(t) \label{Schrodinger}
\end{align}

\begin{remark}
    L'équation d'évolution est linéaire ; elle est donc en particulier compatible avec le principe de superposition du premier postulat.
\end{remark}

Arrêtons-nous l'espace d'un instant pour nous poser une question. Jusque là, nous avons vu des équations linéaires en le temps: mieux, nous faisons face à des équations simples à résoudre. En partant de ce constat, il serait légitime de se demander comment; à partir d'équations linéaires, émergent les propriétés hautement non-linéaires rencontrées parfois en mécanique classique. Un bon exemple est les systèmes chaotiques.\\

Il se trouve que toute la complexité d'un système est encodée dans le spectre de l'Hamiltonien du système.

\section{Quatrième postulat: Interprétation de Bohr}



\part{Symétries en mécanique quantique}

\chapter{Description générale des symétries}

\section{Pourquoi étudier les symétries ?}

Les symétries jouent un rôle fondamental en physique, et la mécanique quantique n'y fait pas exception. Elles permettent d'expliquer les dégénérescences des valeurs propres des observables, et peuvent expliquer certaines propriétés intrinsèques des particules, comme par exemple le spin qui est une conséquence de la symétrie sous les rotations.

Pour illustrer cela, considérons un oscillateur harmonique à deux dimensions non-couplé, de masses identiques. Le hamiltonien du système s'écrit donc

\begin{equation}
  H = \frac{p_1^2}{2m} + \frac{1}{2}m\omega^2x_1^2 + \frac{p_2^2}{2m} + \frac{1}{2}m\omega^2x_2^2
\end{equation}

On peut construire les états propres de $H$ à l'aide des opérateurs création comme dans le cas à une dimension, nous avons donc les états 

\begin{equation}
  \ket{n_1, n_2} \equiv (a_1^\dagger)^{n_1} (a_2^\dagger)^{n_2} \ket{0}
\end{equation}

Les valeurs propres de $H$ sont

\begin{align}
  E_{n_1, n_2} &= \hbar\omega(n_1 + n_2 + 1) \\
  &= \hbar\omega (N+1)
\end{align}

Il est évident que la valeur propre $E_N$ est $N+1$ fois dégénérée. Cette dégénérescence du spectre vient en réalité d'une symétrie de $H$. Quelle symétrie serait susceptible de la provoquer?

\begin{itemize}
    \item Invariance par translation temporelle ? Oui, mais étant vraie pour tout système physique isolé (sans échange d'énergie), elle ne provoque pas la dégénérescence observée.
    \item Invariance par rotation dans les plans $(p_1p_1)$ et $(x_1x_2)$ ? En effet, les termes $(p_1^2 + p_2^2)$ et $(x_1^2 + x_2^2)$ apparaissent pour les impulsions et pour les positions, ce qui induit une dépendance radiale dans ces plans. Cependant, les matrices représentant ces rotations appartiennent à $SO(2)$, qui est un groupe abélien. Cela n'implique alors pas de dégénérescense du spectre (Pourquoi ?)
\end{itemize}

Ces symétries évidentes n'expliquent pas la dégénérescence du spectre, il existe donc une symétrie cachée au problème !


Réécrivons le hamiltonien $H = \hbar\omega(a_1^\dagger a_1 + a_2^\dagger a_2 + 1)$. Si nous appliquons une transformation des opérateurs $a'_i = U_i^j a_j$, avec $U^\dagger U = \mathds{1}$, le Hamiltonien reste inchangé. La propriété de la matrice $U$ Nous indique donc que le hamiltonien est invariant sous l'action de $U(2)$, groupe des matrices unitaires $2\times 2$.

Le sous-espace associé à la valeur propre $\hbar\omega(N+1)$ est donc une représentation du groupe $SU(2)$ de spin $N/2$, dont la dimension\footnotemark coïncide avec le degré de dégénérescence de cette valeur propre.

\footnotetext{Pour rappel, la dimension d'une représentation de SU(2) de spin $j$ est de dimension $2j+1$.}

\section{Discussion générale}

Comment décrit-on les symétries en mécanique quantique? Une symétrie est intuitivement un endomorphisme de l'espace des états $\mathcal{E}$, qui doit préserver la physique décrite, autrement dit le spectre des observables.

\begin{definition}
Un endomorphisme $\mathcal{U}:\mathcal{E} \to \mathcal{E}$ est une \emph{symétrie} si et seulement si $\forall A \in \mathcal{L}(\mathcal{H})$, $\mathcal{U}(A)$ possède le même spectre que $A$. En général,
$$ \tr(\rho A) = \tr(\mathcal{U}(\rho)\mathcal{U}(A)) $$
\end{definition}

Une conséquence de cette propriété fondamentale des symétries est le théorème de Wigner, qui permet de faire le lien entre la symétrie et sa représentation sous forme d'opérateur.

Pour les besoins du théorème, introduisons d'abord les notions d'opérateur \emph{anti-linéaire} et \emph{anti-unitaire}.

\begin{definition}
  Un opérateur \emph{anti-linéaire} est un opérateur qui vérifie la propriété
  $$ A(\alpha\phi + \beta\chi) = \alpha^* A\phi + \beta^* A\chi $$ \\
  Un opérateur \emph{anti-unitaire} est un opérateur qui vérifie la propriété
\end{definition}

\begin{theorem}[Wigner, 1931]
Pour toute symétrie $\mathcal{U}$ dans $\mathcal{E}$, il existe un opérateur $U$ linéaire unitaire, ou anti-linéaire anti-unitaire tel que $\mathcal{U}(\rho) = U\rho U^{-1}$
\end{theorem}


\end{document}
\documentclass{report}
\usepackage[allcolors=true]{hyperref}
\usepackage{graphicx}
\usepackage{amsmath,amsfonts,amsthm,amssymb}
\usepackage{mathrsfs}
\usepackage{wasysym}
\usepackage{caption}
\usepackage{multirow}
\usepackage{physics}
\usepackage{tikz}
\usepackage{cleveref}
\usepackage{pgfplotstable}
\usepackage{siunitx}
\usepackage{wrapfig}
\usepackage{graphicx}
\usepackage{subfiles}
\usepackage{systeme}
\usepackage{bm}
\usepackage{xcolor}
\usepackage[french]{babel}
\usepackage{titlesec}
\usepackage{lmodern}
\usepackage{braket}
\usepackage{calrsfs}
\usepackage{chngcntr}
\usepackage{textcomp}
\usepackage{geometry}
\geometry{top=3cm, bottom=3cm, left=2.5cm, right=2.5cm}

\numberwithin{equation}{part}
\counterwithin*{section}{part}

\pgfplotsset{width=10cm,compat=1.16}

\newcommand{\ti}{\times}
\newcommand{\h}{\hbar}
\renewcommand{\d}{\mathrm{d}}
\renewcommand{\thepart}{\arabic{part}}


\titleformat{\paragraph}
{\normalfont\normalsize\bfseries}{\theparagraph}{1em}{}
\titlespacing*{\paragraph}
{0pt}{3.25ex plus 1ex minus .2ex}{1.5ex plus .2ex}

\title{\textbf{PHYS-F302 - Mécanique Quantique} \\ \textit{Basé sur le cours théorique de Prof. $\href{mailto:frank.ferrari@ulb.be}{\text{Frank Ferrari}}$}}
\author{$\href{mailto:juian.moeil@ulb.be}{\text{Moeil Juian}}$}
\date{\textbf{Année académique 2021-2022}}

\newtheorem{theorem}{Théorème}[section]
\newtheorem{definition}[theorem]{Définition}
\newtheorem{lemma}[theorem]{Lemme}
\newtheorem{Property}[theorem]{Proposition}
\newtheorem{corollary}[theorem]{Corollaire}
\newtheorem{remark}[theorem]{Remarque}
\newtheorem*{preuve}{Preuve}
\newtheorem{reminder}[theorem]{Rappel théorique}
\newtheorem{exemple}[theorem]{Exemple}

\renewcommand\qedsymbol{$\blacksquare$}

\setcounter{tocdepth}{2}

\usepackage{enumitem}

%%% Début ajouts Sami
% Definition boite grise avec couleur grise définie
\definecolor{BGgris}{RGB}{222,230,230}
\newcommand\bg[2]{
\begin{center}
\fcolorbox{white}{BGgris}{\parbox{.9\linewidth}{\begin{large} \textit{#1} \end{large} \\

#2 }}
\end{center}}
% Boite d'alerte
\definecolor{BGorange}{RGB}{255, 216, 154}
\newcommand\probleme[1]{
\begin{center}
\fcolorbox{white}{BGorange}{\parbox{.9\linewidth}{\begin{large} \textbf{Problème} \end{large} \\

#1}}
\end{center}}

\setlength{\fboxsep}{2em} % espace entre le bord d'une boite et le texte dedans

%%% Fin ajouts Sami

\begin{document}

\maketitle

\begin{abstract}
    Ces notes sont basées sur les cours et les notes de Frank Ferrari. Dans la mesure où elles sont été rédigées par des étudiant.e.s, il est probable qu'elles contiennent des fautes. N'hésitez pas à les signaler à $\href{mailto:juian.moeil@ulb.be}{\text{Juian Moeil}}$ lorsque vous en trouvez.
\end{abstract}
\newpage

\part{Fondements}

\chapter{Postulats de la Mécanique Quantique}

\section{Premier postulat : description des états quantiques}

Dans la description classique de la Mécanique, on définit un espace de phase $M$, de dimension $2n$, $\forall n\in\mathbb{Z}$; permettant ainsi d'introduire les coordonnées $\left(q^i,p_j\right)$ liées par le crochet de poisson fondamental $\{q^i,p_j\}=\delta^i_j$. Un état est caractérisé par un point P dans l'espace des phases.\\

En Mécanique Quantique, on définit un état comme une raie dans un espace de Hilbert $\mathcal{H}$. En introduisant la relation d'équivalence
\begin{equation}
\label{equivalence}
    \ket{\psi} \sim \ket{\phi} \iff \exists \alpha\in\mathbb{C}\setminus\{0\} \text{tel que} \ket{\psi}=\alpha\ket{\phi} ;
\end{equation}
nous pouvons définir l'espace des états comme étant $\mathcal{H}\setminus\ref{equivalence}$. Les raies sont les classes d'équivalence, et $\ket{\psi}$ et $\ket{\phi}$ représentent le même état.\\

\begin{Property}
    Puisque $\mathcal{H}$ est un espace vectoriel, nous avons que pour deux états $\ket{\psi},\ket{\phi} \in\mathcal{H}$, 
    \begin{equation}
        \alpha\ket{\psi}+\beta\ket{\psi}=\ket{\varphi}
    \end{equation}
    est également un état de $\mathcal{H}$. Il s'agit du principe de superposition : toute combinaison linéaire d'état donne un autre état dans la base de $\mathcal{H}$ilbert.
    \label{Superposition}
\end{Property}

\section{Second postulat: observables $\ket{\hat{P}}$}

\begin{definition}
    Un opérateur est hermitien lorsque $A=A^\dagger$.
\end{definition}

\begin{definition}
    Un opérateur hermitien est dit observable lorsqu'il possède une base de vecteurs propres.
\end{definition}

\begin{remark}
    Le théorème spectral nous indique alors que le spectre $A=A^\dagger$ est réel. Notons également que le spectre peut-être discret et/ou continu. 
\end{remark}

\begin{remark}
    Une conséquence de cet résultat est la non commutativité des opérateurs. De manière générale, nous avons que $[A,B] \neq 0$.
\end{remark}



\end{document}